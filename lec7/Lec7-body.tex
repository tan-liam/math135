% !TEX root = Lec7.tex

\section{Equivalence Relations}
\begin{rem}
The main topic for today is the notion of an equivalence relation. They are used to make sense of calling two things
the same when they are not literally equal, for whatever intents and purposes in whatever particular case. Part of the
goal of this lecture is to turn equivalence relative to an equivalence relation into literal equality through application
of some function.

\end{rem}
\begin{defn}[Equivalence Relation]
$E$ will be a relation. $E$ is an equivalence relation (on a set $X$) if
\begin{enumerate}
    \item $X = \fld(E)$.
    \item $E$ is reflexive. i.e. $\forall a \in X, \la a, a\ra \in E$. Equivalently, $I_X \subseteq E$.
    \item $E$ should be symmetric. i.e. $\forall a \forall b, \la a,  b\ra \in E \implies \la b, a \ra \in E$. Equivalently, $E^{-1} \subseteq E$.
    \item $E$ is a transitive relation. i.e. $\forall a \forall b \forall c, \la a, b\ra \in E, \la b, c\ra \in E \implies \la a, c\ra \in E$. Equivalently, $E \circ E \subseteq E$.
\end{enumerate}
\end{defn}
\begin{prop}
A relation $R$ is transitive if and only if $R \circ R \subseteq R$.
\end{prop}
\begin{proof}
Suppose $R \circ R \subseteq R$. Let $\la a, b\ra \in R$ and $\la b, c\ra \in R$. We know that by the definition
of composition, $\la a, c\ra \in R \circ R \subseteq R$.\\
$\impliedby$. Suppose $\la a, b\ra \in R$ and $\la b, c\ra \in R$. Then we know by transitivity of the relation that 
$\la a, c\ra \in R$. Suppose $\la a, c\ra \in R \circ R$. Then there exists $a, b, c$ such that 
$\la a, b\ra \in R$ and $\la b, c\ra \in R$. By transitivity, $\la a, c\ra \in R$.
\end{proof}
\begin{prop}
A relation $R$ is reflexive iff $I_X \subseteq R$, where $X = \fld(R)$
\end{prop}
\begin{prop}
A relation $R$ is symmetric iff $R^{-1} \subseteq R$ ($R^{-1} = R$)
\end{prop}
\begin{ex}
$X$ is the set of formulas in some language.
\[R = \{\la \psi, \varphi \ra: \psi \vdash \varphi\}\]
This is reflexive and transitive, but not symmetric.
\end{ex}
\begin{ex}
    Let $X$ be any set. 
    \[R = I_X\]
    This is an equivalence relation.
\end{ex}
\begin{ex}
    Let $X = \Z$. Fix $n \in \Z_+$
    \[E_n = \{\la a, b\ra \in \Z \times \Z: n | (a - b)\}\]
    This is an equivalence relation. Feel free to do the computation to check.
\end{ex}
\begin{prop}
Let $f: X \to Y$. 
\[E_f = \{\la a, b\ra \in X \times X: f(a) = f(b)\}\]
is an equivalence relation.
\end{prop}
\begin{proof}
Consider $a \in X$. Then $f(a) = f(a)$, so $\la a, a\ra \in E_f$. Let $b \in X$ be such that $\la a, b\ra \in E_f$.
Then $f(a) = f(b) \implies f(b) = f(a)$, so $\la b, a\ra \in E_f$. Finally, let $\la a, b\ra \in E_f$ and let $\la b, c\ra \in E_f$.
Then $f(a) = f(b)$ and $f(b) = f(c)$, so $f(a) = f(c) \implies \la a, c\ra \in E_f$.
\end{proof}
\begin{ex}
Consider the above example
\[E_n = \{\la a, b\ra \in \Z \times \Z: n | (a - b)\}\]
\[\text{rem}_n : \Z \to Z\]
\[m \mapsto m \% n\]
\[E_{\text{rem}_n} = E_n\]
\end{ex}
\begin{defn}[E-equivalence Class]
The $E$-equivalence class of $x \in X$ is
\[[x]_E = \{y \in X: \la x, y\ra \in E\}\]
\end{defn}
\begin{prop}
We will construct $f: X \to \P(X)$ 
\[f := \{t \in X \times \P(X): \exists x \exists A, x \in X, A \in \P(X), t = \la x, A\ra, A = \{y \in X: \la x, y \ra \in E\}\}\]
Claim:
\[E = E_f\]
\end{prop}
\begin{proof}
Let $\la a, b\ra \in E$. Consider 
\[f(a) = \{y \in X: \la a, y\ra \in E\}\]
\[f(b) = \{y \in X: \la b, y\ra \in E\}\]
Consider $y \in f(a)$. Then $\la a, y\ra \in E$. Since $\la a, b\ra \in E$, by symmetry we have that $\la b, a\ra \in E$. By 
transitivity, $\la b, y\ra \in E$, so $y \in f(b)$. Thus $f(a) \subseteq f(b)$. By similar logic, we can show that $f(b)\subseteq f(a)$.
Thus 
\[f(a) = f(b) \implies \la a, b\ra \in E_f\]
$\impliedby$ Let $\la a, b\ra \in E_f$. $f(a) = f(b)$. Let $y \in f(a)$. Then $\la a, y\ra \in E$. We also know that 
$y \in f(b)$. Then $\la b, y\ra \in E$. By symmetry, $\la y, b\ra \in E$. Thus by transitivity, $\la a, b\ra \in E$.
Thus $E_f \subseteq E$
\end{proof}
\begin{prop}
\[\dom f = X\]
\end{prop}
\begin{proof}
Let $x \in X$. $A = \{y \in X: \la x, y\ra \in E\}$. Then $\la x, A\ra \in t$. Thus 
\[X \subseteq \dom f \subseteq X\]
\end{proof}
\begin{prop}
    $f$ is a function
    \end{prop}
    \begin{proof}
    Let $x \in X$. Consider $\la x, A\ra \in f$ and $\la x, B\ra \in f$. Then 
    \[A = \{y \in X: \la x, y\ra \in E\}\]
    \[B = \{z \in X: \la x, z\ra \in E\}\]
    By extensionality, clearly $A = B$.
    \end{proof}
\begin{rem}
We started with $E$, an equivalence relation on a set $X$ and constructed 
\[f: X \to \P(X)\]
\[a \mapsto \text{E-equivalence class of } a =  \{y \in X: \la a, y\ra \in E\}\]
We showed that $E = E_f$. A question we might ask is what is the range of $f$?
\end{rem}
\begin{defn}[Quotients]
\[X/E = \{A \in P(X): \exists a, a \in X, A = \{y \in X: \la a, y\ra \in E\}\}\]
\[X/E = \ran f\]
This is the set of $E$-equivalence classes.
\end{defn}
\begin{defn}[Canonical Mapping]
We will change notation to 
\[\pi_E: X \to X/E\]
\[a \mapsto [a]_E = \{y \in X: \la a, y\ra \in E\}\]
This is called the canonical mapping
\end{defn}
