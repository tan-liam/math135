% !TEX root = Lec2.tex

\section{Introduction to the Theory of Sets}

We will formalize set theory and prove theorems based on the constructions from the axioms. The theory we will be developing is called ZFC (Zermelo-Fraenkel Set Theory with Choice). ZFC is a first-order logical theory expressed in the language of set theory called \[\mathscr{L}(\epsilon)\]
$\epsilon$ is the only non-logical symbol, called a binary relation, "$a \in b$".
\\\\
Where $a$ and $b$ are variables, we can combine atomic formula, logical formula, and quantifiers to form our language. Atomic formula include $a = b$ and $a \in b$. Logical formula include $a \implies b$, $\neg a$, $a \wedge b$, $a \vee b$, $a \iff b$. Quantifiers are $\exists x \varphi$ and $\forall x \varphi$.
\\\\
An $\mathscr{L}(\epsilon)$-structure is a nonempty set V together with a set $\epsilon \subset V \times V$. For $a, b \in V$
\[(V, \epsilon^V) \models a \in b \iff (a, b) \in \epsilon^V\]
\begin{ex}[The universe of sets]
\[(\mathbb{V}, \epsilon)\]
\end{ex}
\begin{thm} [ZFC cannot prove that there is a model of ZFC] i.e. a set V and a set $\epsilon^V \subseteq V \times V$ s.t every axiom of ZFC is true.
\end{thm}
\begin{thm}[Godel's Incompleteness Theorem for ZFC] If ZFC can prove that ZFC is consistent then ZFC is inconsistent 
\end{thm}
\subsection{Introducing the Axioms}
\begin{axiom}
    [Axiom of Extensionality] This axiom defines what it means for two sets to be the same. The equality operator is taken to be primitive in first-order logic and we need to define how it relates to the other symbols.
    \[\forall A \forall B [A = B \iff \forall t [t \in A \iff t \in B]]\]
\end{axiom}
\begin{ex}[An example of a structure]
\[V = \{0, 1\}\]
\[\epsilon^V = \{(0,0), (0, 1)\}\]
\[V \models 0 \in 1\]
\[V \models 0 \in 0\]
$(V, \epsilon^V) \not\models$ the axiom of extensionality.
\[V \models \neg(0 = 1)\]
but 
\[V \models \forall t [t \in 0 \iff t \in 1]\]
\end{ex}
\begin{axiom}
    [Empty Set Axiom] There exists an empty set.
    \[\exists x \forall t \neg[t \in x]\]
    We can abbreviate this with the following
    \[\exists x \forall t (t \notin x)\]
\end{axiom}
\begin{defn}[Defining the $\notin$ symbol]
    We want to define formula that aren't strictly part of our language. Let us define the negation of the set membership.
    \[\Delta_{\notin}: \forall x \forall y [x \notin y \iff \neg(x \in y)] \in \mathscr{L}(\epsilon, \not\epsilon)\]
\end{defn}
\begin{prop}
    For any $\mathscr{L}(\epsilon)$ structure $(M, \epsilon^M)$, there is unique expansion $(M, \epsilon^M, \not\epsilon^M) \models \Delta_{\notin}$
\end{prop}
\begin{rem}[Introducing Symbols] We will want to introduce many new symbols throughout the course of this class. e.g. $\subseteq, \mathcal{P} \mathbb{A}$. We will need to define new symbols logically each time we introduce them and can do so hierarchically with previously defined symbols.
\end{rem}
\begin{ex}[A structure consistent with the Axioms]
\[V = \{0\}\]
\[\epsilon^V = \emptyset\]
$(V, \epsilon^V) \models $ Empty Set and Extensionality Axioms
\end{ex}
\begin{prop}[Uniqueness of the Empty Set]
There is exactly one empty set.
\end{prop}
\begin{proof}
By the Empty Set Axiom, there is at least one empty set. If $x$ and $y$ are two empty sets. i.e. $(\forall t) t \notin x$ and $(\forall t) t \notin y$, then $\forall t (t \in x \iff t \in y)$. By the Axiom of Extensionality, $x = y$
\end{proof}
\begin{defn}[Empty Set Formula]
\[\Delta_\emptyset : \forall x [x = \emptyset \iff \forall t (t \notin x)] \in \mathscr{L}(\epsilon, \not\epsilon, \emptyset)\]
where the $\epsilon$ and $\not\epsilon$ are binary relation symbols and the $\emptyset$ symbol is called a constant symbol.
\end{defn}
\begin{defn}[Subset Formula]
\[\Delta_{\subseteq}: \forall x \forall y [x \subseteq y \iff \forall t [t \in x \implies t \in y]]\]
\end{defn}
\begin{axiom}[Power Set Axiom]
\[\forall x \exists z \forall t [t \in z \iff t \subseteq x]\]    
\end{axiom}
\begin{prop}[Uniqueness of the Power Set]
    $\forall x$ there is a unique power set of $x$
\end{prop}
\begin{defn}[Power Set Operation]
\[\Delta_{\mathcal{P}}: \forall x \forall z [z = \mathcal{P}x \iff \forall t [t \in z \iff t \subseteq x]]\]
$\mathcal{P}$ is a unary function symbol.
\end{defn}
\begin{ex}[What Sets can we build so far]
$\emptyset, \mathcal{P}(\emptyset)= \{\emptyset\}, \mathcal{P}\mathcal{P}\emptyset = \{\emptyset, \{\emptyset\}\}$, etc
We can keep iterating powersets. We will need to add more axioms. For example, we cannot prove the existence of  $\{\{\emptyset\}\}$ with the axioms we have so far.
\end{ex}
\begin{axiom}[Pair Set Axiom]
\[\forall x\forall y \exists z \forall t [t \in z \iff (t = x \vee t= y)]\]
\end{axiom}
\begin{defn}[Pair Set Operation]
\[\Delta_{\{,\}}: \forall x \forall t \forall z [z = \{x, y\} \iff \forall t [t \in z \iff t = z \vee t = y]] \]    
\end{defn}
\begin{ex}[Singleton Set]
\[x = y\]
\[\{x, y\} = \{x, x\} = \{x\}\]
\end{ex}
\begin{prop}[Existence of the Singleton]
\[\forall x \exists z \forall t [t \in z \iff t = x]\]
\end{prop}
\begin{proof}
We can apply the Pair Set Axiom with $y = x$
\end{proof}
\begin{defn}[Singleton Operation]
\[\Delta_{\{\}} = \forall x \forall y [z = \{x\} \iff \forall t [t \in z \iff t = x]]\]
\end{defn}
\begin{ex}[Triplet Set]
Can we create a set with three items? We can attempt to apply the pair set axiom to $\{
a\}$ and $\{b, c\}$, but unfortunately we will end up with $\{\{a\}, \{b, c\}\}$. We will need to define the union.
\end{ex}
\begin{axiom}[Binary Union Axiom (Not ZFC)]
\[\forall x \forall y \exists z \forall t [t \in z \iff (t \in x \vee t \in y)]\]
\end{axiom}
\begin{defn}[Binary Union Operation]
\[\Delta_\cup : \forall x \forall y \forall z [z = x \cup y \iff \forall t [t \in z \iff (t \in x \vee t \in y)]]\]
\end{defn}
\begin{ex}[Triplet Set Reprise]
Now with the binary union axiom, we can apply the operation to $\{a\}$ and $\{b, c\}$, which were found by the pair set axiom. Using the Binary Union Axiom, 
\[\{a\} \cup \{b, c\} = \{a, b, c\}\]
\end{ex}